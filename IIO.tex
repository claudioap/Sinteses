\documentclass[]{report}
\usepackage[a4paper, total={7in, 8.5in}]{geometry}
\usepackage[portuguese]{babel}
\usepackage{hyperref}
\usepackage{graphicx}
\usepackage{amsmath}
\usepackage{amsfonts}

\begin{document}
\begin{titlepage}
	\centering
	\vspace{5cm}
	{\huge\bfseries Introdução à Investigação Operacional\par}
	\vspace{1cm}
	{\scshape\Large Síntese baseada no conteudo lecionado na\\
	 FCT/Universidade Nova de Lisboa\par}
	\vspace{2cm}
	Adaptado por:\\
	{\Large \textit{Cláudio Afonso de Sousa Pereira}\\
	(sinteses$\text{@}$claudiop$.$com)\par}
	\vspace{1cm}
	Do material lecionado por:\\
	{\Large \textit{Ruy Araújo da Costa}\\
	(rcosta$\text{@}$fct$.$unl$.$pt)\par}
	\vspace{1cm}
	{\large \today\par}
	\vfill
	Adaptação licenciada:\\
	\href{http://creativecommons.org/licenses/by-sa/4.0/}{\includegraphics[scale=0.8]{ccbysa.png}}
\end{titlepage}
\chapter{Programação Linear}
\section{Introdução a programação linear}
A programação linear (abreviada PL) consiste na resolução de problemas de maximização (ou minimização) de funções, quando os domínios se encontram restritos por equações de 1º grau (lineares).\\ \par
Nos problemas de PL existe a \textbf{função objetivo} $F$, uma função de 1º grau que se pretende maximizar ou minimizar. Essa função tem as variaveis $x_1, x_2, \dots, x_n$ sendo a sua forma extensa $F=c_1 \cdot x_1 + c_2 \cdot x_2 + \dots + c_n \cdot x_n$.\\
Em F impomos um conjunto de restrições de três tipos, $\leq, \geq, =$.\\
O objetivo da programação linear é a extração de um \textbf{valor ótimo} que se costuma representar $F^\star$.
\subsection{Problema de programação linear}
Os problemas de programação linear tem a seguinte forma:
\begin{quotation}
\noindent
Maximizar (ou minimizar) $F(X_1, X_2, \dots, X_n)$\\
Sujeito a:
$$\begin{array}{rrl}
c_1 \cdot X_1  + & c_2 \cdot X_2 & + \dots + c_n \cdot X_n \quad \{\leq, \geq, =\}\quad b_1\\
c_{n+1} \cdot X_1 + & c_{n+1} \cdot X_2 & + \dots + c_{2n} \cdot X_n \quad \{\leq, \geq, =\}\quad b_2
\end{array}
$$
\end{quotation}
\subsection{Formulação de problema}
A titulo de exemplo, queiram-se vender dois produtos $A$ e $B$ maximizando o lucro.\\
$A$ vende-se por 10 batatas, e para fabricar necessita de 3 porcas e 2 parafusos.\\
$B$ vende-se por 12 batatas, e para fabricar necessita de 5 porcas e 1 parafuso.\\ 
Dispõe-se de 100 porcas e 30 parafusos.\\
As porcas e os parafusos são gratuitos.\\ \\
A função objetivo é $F = 10 \cdot A + 12 \cdot B$ e quer-se \textbf{maximizar}.\\
Esta função está sujeita a:
$$\begin{array}{rccll}
3 \cdot A &+& 5 \cdot B &\leq 100 \quad &\text{Restrição das porcas}\\
2 \cdot A &+& 1 \cdot B &\leq 30 \quad &\text{Restrição dos parafusos}\\
&&A, B& \geq 0& \text{Positividade das variáveis}
\end{array}
$$
Este problema está formulado na \textbf{forma geral} (ver abaixo).
\section{Formas de problemas}
Os problemas de PL podem estar formulados de formas distintas.
\subsection{Forma geral}
A \textbf{forma geral} é a forma obtida quando se formulam os problemas.
\begin{itemize}
\item A função objetivo podem ser maximizações ou minimizações.
\item As restrições podem ser dos tipos $\leq, \geq$ ou $=$.
\item As variáveis podem ser $\geq 0$, $\leq 0$ ou $\in \mathbb{R}$.
\end{itemize}
\subsection{Forma canónica}
A \textbf{forma canónica} impõe restrições sobre a forma geral.
\begin{itemize}
\item A função objetivo só pode ser uma maximização.
\item As restrições tem de ser todas majorações ($cX + \dots  \leq k$).
\item As variáveis são positivas ou nulas ($\geq 0$).
\end{itemize}
Efetuar a passagem de um problema da forma geral para a forma canónica é feito com as transformações:
\begin{itemize}
\item (Caso objetivo seja minimização) Maximização de uma função que seja simétrica à função objetivo.\\
Se $F(X)=-G(X)$ então a minimização de $F$ é a maximização de $G$.\\
Neste caso o valor ótimo $F^\star$ é dado por $-G^\star$
\item (Caso restrições negativas) Multiplicar a restrição por $-1$.\\
$cX \geq k \quad \Rightarrow \quad -cX \leq k$
\item (Caso variáveis $\notin \mathbb{R}^+_0$) Substituir uma variável $x$ por $y-y'$.\\
Se $x \leq 0$ então $y,y' \geq 0$ conseguem atingir todos os valores que $x$ atingia.\\
Não esquecer de substituir em todos os locais em que a variável em questão aparece.
\end{itemize}
\subsection{Forma standard}
A \textbf{forma standard} baseia-se na forma canónica impondo ainda mais restrições na forma geral. É uma formulação melhor para resoluções analíticas (como será visto).\\[0.5cm]
A única diferença da forma canónica é que a standard requer que as restrições sejam igualdades. Para tal requer-se a introdução de um novo conceito:
\paragraph{Variáveis de folga} - Assumem a diferença entre a soma das variáveis de uma restrição e a constante que as limita.\\
A aplicação das mesmas é feita com a transformação $cX \leq k \quad \Rightarrow \quad cX + f = k$ com $f$ sendo a variável de folga.\\[0.5cm]
\noindent A passagem da forma canónica para a standard passa pela colocação de variáveis de folga em todas as restrições.
\subsection{Sintetização}
\begin{table}[ht!]
\begin{tabular}{l|l|l|l}
                & Geral           & Canónica & Standard \\\hline
Função objetivo & max/min         & max      & max \\\hline
Restrições      & $\geq,\leq, =$  & $\leq$   & $=$\\\hline
Variáveis       & $\geq 0,\leq 0,\in \mathbb{R}$ & $\geq 0$ & $\geq 0$         
\end{tabular}
\end{table}
\chapter{Teoria da Decisão}
\chapter{Teoria das Filas de Espera}
\chapter{Simulação}
\chapter{Programação Linear}
\end{document}
