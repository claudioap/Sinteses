\documentclass[]{report}
\usepackage[a4paper, total={7in, 8.5in}]{geometry}
\usepackage[portuguese]{babel}
\usepackage{hyperref}
\usepackage{graphicx}
\usepackage{amsmath}
\usepackage{amsfonts}
\usepackage{forest}
\usepackage{tikz}
\tikzset{
  treenode/.style = {align=center},
  root/.style     = {treenode, font=\Large},
  env/.style      = {treenode, font=\normalsize},
  dummy/.style    = {circle,draw}
}

\begin{document}
\begin{titlepage}
	\centering
	\vspace{5cm}
	{\huge\bfseries Introdução à Investigação Operacional\par}
	\vspace{1cm}
	{\scshape\Large Síntese baseada no conteúdo lecionado na\\
	 FCT/Universidade Nova de Lisboa\par}
	\vspace{2cm}
	Adaptado por:\\
	{\Large \textit{Cláudio Afonso de Sousa Pereira}\\
	(sinteses$\text{@}$claudiop$.$com)\par}
	\vspace{1cm}
	Do material lecionado por:\\
	{\Large \textit{Ruy Araújo da Costa}\\
	(rcosta$\text{@}$fct$.$unl$.$pt)\par}
	\vspace{1cm}
	{\large \today\par}
	\vfill
	Adaptação licenciada:\\
	\href{http://creativecommons.org/licenses/by-sa/4.0/}{\includegraphics[scale=0.8]{ccbysa.png}}
\end{titlepage}
\chapter{Programação Linear}
\section{Introdução a programação linear}
A programação linear (abreviada PL) consiste na resolução de problemas de maximização (ou minimização) de funções, quando os domínios se encontram restritos por equações de 1º grau (lineares).\\ \par
Nos problemas de PL existe a \textbf{função objetivo} $F$, uma função de 1º grau que se pretende maximizar ou minimizar. Essa função tem as variaveis $x_1, x_2, \dots, x_n$ sendo a sua forma extensa $F=c_1 \cdot x_1 + c_2 \cdot x_2 + \dots + c_n \cdot x_n$.\\
Em F impomos um conjunto de restrições de três tipos, $\leq, \geq, =$.\\
O objetivo da programação linear é a extração de um \textbf{valor ótimo} que se costuma representar $F^\star$.
\subsection{Problema de programação linear}
Os problemas de programação linear tem a seguinte forma:
\begin{quotation}
\noindent
Maximizar (ou minimizar) $F(X_1, X_2, \dots, X_n)$\\
Sujeito a:
$$\begin{array}{rrl}
c_1 \cdot X_1  + & c_2 \cdot X_2 & + \dots + c_n \cdot X_n \quad \{\leq, \geq, =\}\quad b_1\\
c_{n+1} \cdot X_1 + & c_{n+1} \cdot X_2 & + \dots + c_{2n} \cdot X_n \quad \{\leq, \geq, =\}\quad b_2
\end{array}
$$
\end{quotation}
\subsection{Formulação de problema}
A titulo de exemplo, queiram-se vender dois produtos $A$ e $B$ maximizando o lucro.\\
$A$ vende-se por 10 batatas, e para fabricar necessita de 3 porcas e 2 parafusos.\\
$B$ vende-se por 12 batatas, e para fabricar necessita de 5 porcas e 1 parafuso.\\ 
Dispõe-se de 100 porcas e 30 parafusos.\\
As porcas e os parafusos são gratuitos.\\ \\
A função objetivo é $F = 10 \cdot A + 12 \cdot B$ e quer-se \textbf{maximizar}.\\
Esta função está sujeita a:
$$\begin{array}{rccll}
3 \cdot A &+& 5 \cdot B &\leq 100 \quad &\text{Restrição das porcas}\\
2 \cdot A &+& 1 \cdot B &\leq 30 \quad &\text{Restrição dos parafusos}\\
&&A, B& \geq 0& \text{Positividade das variáveis}
\end{array}
$$
Este problema está formulado na \textbf{forma geral} (ver abaixo).
\section{Formas de problemas}
Os problemas de PL podem estar formulados de formas distintas.
\subsection{Forma geral}
A \textbf{forma geral} é a forma obtida quando se formulam os problemas.
\begin{itemize}
\item A função objetivo podem ser maximizações ou minimizações.
\item As restrições podem ser dos tipos $\leq, \geq$ ou $=$.
\item As variáveis podem ser $\geq 0$, $\leq 0$ ou $\in \mathbb{R}$.
\end{itemize}
\subsection{Forma canónica}
A \textbf{forma canónica} impõe restrições sobre a forma geral.
\begin{itemize}
\item A função objetivo só pode ser uma maximização.
\item As restrições tem de ser todas majorações ($cX + \dots  \leq k$).
\item As variáveis são positivas ou nulas ($\geq 0$).
\end{itemize}
Efetuar a passagem de um problema da forma geral para a forma canónica é feito com as transformações:
\begin{itemize}
\item (Caso objetivo seja minimização) Maximização de uma função que seja simétrica à função objetivo.\\
Se $F(X)=-G(X)$ então a minimização de $F$ é a maximização de $G$.\\
Neste caso o valor ótimo $F^\star$ é dado por $-G^\star$
\item (Caso restrições negativas) Multiplicar a restrição por $-1$.\\
$cX \geq k \quad \Rightarrow \quad -cX \leq k$
\item (Caso variáveis $\notin \mathbb{R}^+_0$) Substituir uma variável $x$ por $y-y'$.\\
Se $x \leq 0$ então $y,y' \geq 0$ conseguem atingir todos os valores que $x$ atingia.\\
Não esquecer de substituir em todos os locais em que a variável em questão aparece.
\end{itemize}
\subsection{Forma standard}
A \textbf{forma standard} baseia-se na forma canónica impondo ainda mais restrições na forma geral. É uma formulação melhor para resoluções analíticas (como será visto).\\[0.5cm]
A única diferença da forma canónica é que a standard requer que as restrições sejam igualdades. Para tal requer-se a introdução de um novo conceito:
\paragraph{Variáveis de folga} - Assumem a diferença entre a soma das variáveis de uma restrição e a constante que as limita.\\
A aplicação das mesmas é feita com a transformação $cX \leq k \quad \Rightarrow \quad cX + f = k$ com $f$ sendo a variável de folga.\\[0.5cm]
\noindent A passagem da forma canónica para a standard passa pela colocação de variáveis de folga em todas as restrições.
\subsection{Sintetização}
\begin{tabular}{l|l|l|l}
\textbf{Forma}:  & Geral           & Canónica & Standard \\\hline
Função objetivo: & max/min         & max      & max \\\hline
Restrições:      & $\geq,\leq, =$  & $\leq$   & $=$\\\hline
Variáveis:       & $\geq 0,\leq 0,\in \mathbb{R}$ & $\geq 0$ & $\geq 0$         
\end{tabular}
\section{Soluções}
A \textbf{solução} de um problema de PL é qualquer combinação de coeficientes que cumpra todas as restrições.\\
Uma solução é ótima se obtiver a parir de $F$ o valor ótimo, no entanto obter um valor ótimo não é requisito para ser solução.\\
Se uma solução obedecer ao domínio das variáveis é \textbf{solução admissível}, se sair fora desse domínio (continuando a cumprir as restrições) é \textbf{solução não admissível}.\\
As variáveis cujos coeficientes são não-nulos chama-se de \textbf{básicas}. [contradito pelas degeneradas, confirmar]\\[0.5cm]
Fale-se de um problema na forma standard com $n$ variáveis e $m$ restrições ($n>m$):
\begin{itemize}
\item \textbf{Solução básica} é qualquer solução $(n-m)$ variáveis básicas.\\
Isto é, tantas variáveis básicas quão variáveis de folga.
\item \textbf{Base} é o conjunto das variáveis básicas de uma solução.
\item \textbf{Degenerada} é a solução admissível que contem variáveis básicas nulas.\\
Soluções degeneradas ocorrem nas intersecções de mais de duas restrições.
\end{itemize}
\begin{tikzpicture}
  [
    every node/.style={circle,draw},
    level 1/.style={sibling distance=20em},
    level 2/.style={sibling distance=20em},
    level 3/.style={sibling distance=10em},
    grow                    = down,
    level distance          = 7em,
    edge from parent/.style = {draw, -latex},
    every node/.style       = {font=\small},
    sloped
  ]
  \node [root] {Tuplo coeficientes}
    child { node [env] {Não é solução}
      edge from parent node [below] {Restrições falham} }
    child { node [dummy] {}
      child { node [dummy] {}
        child { node [env] {Solução\\ básica\\ admissivel\\ (S.B.A.)}
                edge from parent node [above] {Básica} }
        child { node [env] {Solução\\ não básica\\ admissivel\\ (S.N.B.A.)}
                edge from parent node [above] {Não básica} }
        edge from parent node [below] {Admissível} }
      child { node [dummy] {}
        child { node [env] {Solução\\ básica\\ não admissivel\\ (S.B.N.A)}
                edge from parent node [above] {Básica} }
        child { node [env] {Solução\\ não básica\\ não admissivel\\ (S.N.B.N.A)}
                edge from parent node [above] {Não básica} }
        edge from parent node [below] {Não admissível} }
              edge from parent node [below] {Restrições passam} };
\end{tikzpicture}
\subsection{Integralidade}
Um problema de programação linear pressupõe coeficientes em $\mathbb{R}$, no entanto pode ser importante obter coeficientes inteiros. Ninguém quer comprar meio parafuso. Esses problemas requerem que o domínio das suas variáveis seja intersectado ao conjunto dos inteiros $\mathbb{Z}$, passando a chamar-se de problemas de programação linear inteira (abreviado PLI).\\ \par
A solução de um problema de PLI é sempre $\leq$ á solução do mesmo problema em PL (com domínios em $\mathbb{R}$).\\
Note-se ainda que a solução de um problema PLI \underline{não é} necessariamente o arredondamento da solução do mesmo problema em PL. A coordenada inteira admissível que mais se aproxima da reta de ótimalidade pode estar a mais de uma unidade de distância do da coordenada real ótima.\\[0.5cm]
Ao problema de PL resultante do levantamento das restrições de integralidade de um problema PLI, chama-se de "problema relaxado", pelo que "relaxar" um PLI é precisamente retirar o critério de integralidade.
\subsection{Procura de soluções ótimas}
Na resolução de um problema de PL considera-se o facto que todas as soluções (quando finitas) serão vértices.\\
Existem diversos algoritmos para a procura de soluções ótimas, sendo a solução menos elaborada a enumeração dos vértices, com respetivos valores de $F$ para se concluir a ótimalidade.\\
No entanto é possível mesmo sem algoritmos concluir alguns dados através do numero de variáveis e restrições.\\[0.5cm]
Tenha um problema de PL na forma standard $n$ variáveis e $m$ restrições:\\
Uma solução é básica caso se encontre num vértice, e cada restrição poderá causar a aparição de 0, 1 ou 2 vértices. [ilustrar]\\
É possível definir um limite superior para as variáveis básicas admissíveis, sendo: $C^n_m = \frac{n!}{m! / (n-m)!}$. [formula combinatória mal na sebenta?]
\subsection{Teorema fundamental da programação linear}
Dado um problema na forma standard:\\
Se existe uma solução admissível, existe uma solução básica admissível.\\
Se existe uma solução ótima admissível, existe uma solução básica ótima admissível.
\subsection{Múltiplas soluções}
Quando de um problema de PL resulta um segmento de reta de soluções ótimas, descobrem-se quais os vértices da região admissível que fazem extremo a esse segmento de reta.\\
Procede-se então a representar a solução ótima com a expressão $(X^\star, Y^\star) = \lambda (x_1, y_1)+(1-\lambda)(x_2, y_2)$, em que $(x_1, y_1)$ e $(x_2, y_2)$ são os extremos do segmento de reta, e $\lambda$ é um valor limitado entre 0 e 1 (formalmente $\lambda \in [0,1]$).\\
Tem-se assim, variando $\lambda$, todos os valores ótimos.
\section{Método gráfico de resolução}
Em problemas de programação linear de duas variáveis pode ser utilizada uma abordagem gráfica.\\
A abordagem consiste no desenho das restrições num referencial.
\begin{itemize}
\item Cada restrição é colocada por base numa reta.\\
Note-se que caso a restrição inclua desigualdade, então um dos semi-planos limitado pela reta é região admissível.
\item A região admissível é a interceção das sub-regiões admissíveis (uma por cada restrição).
\item Os valores possíveis da função objetivo são retas paralelas entre si.
\item O valor ótimo (caso exista) será um vértice, reta ou segmento de reta.
\end{itemize}
[Por ilustrar e acabar]
\section{Notação matricial}
\begin{minipage}{\textwidth}
Existe uma notação frequentemente utilizada nos problemas de programação linear, a notação matricial, que consiste na formulação do problema na forma de matrizes.\\[0.5cm]
Seja o seguinte um problema com $n$ variáveis e $m$ restrições:
\begin{quotation}
\noindent Maximizar $F=C \cdot X$\\
Sujeito a: $A \cdot X = b$\\
\indent\indent\indent $X \geq 0$
\end{quotation}
Então:
\begin{itemize}
\item $X$ é um vetor coluna com as $n$ variáveis.
\item $C$ o vetor linha dos $n$ coeficientes da função objetivo.
\item $A$ a matriz ($m \times n$) dos coeficientes das restrições.
\item $b$ é o vetor coluna dos $m$ termos independentes das restrições.
\end{itemize}
\end{minipage}
\section{Método simplex primal}
O método \textbf{simplex primal} é um método iterativo que efetua procura de soluções ótimas de um problema de PL iterativamente. O seu funcionamento visto geometricamente passa por saltar entre vértices adjacentes. Como o espaço de soluções é um polítopo, se cada salto representar uma melhoria no valor de $F$, então eventualmente alcança-se o valor ótimo.\\
Começa-se por executar dois importantes passos:
\begin{itemize}
\item Escreve-se o problema na forma standard.
\item Descobre-se uma solução básica admissível inicial.\\
Tipicamente a solução básica admissível inicial tem as variável de folga na base, mas nem sempre isso é possível.\\
\textit{Nota}: Esta solução inicial recomendada corresponde à origem do referencial.
\end{itemize}
\noindent De um problema:
\begin{quotation}
\noindent Maximizar $F=v_1\cdot x +v_2\cdot y$\\
Sujeita a:
\begin{quotation}
\noindent $c_1\cdot x+c_2\cdot y + f_1 = b_1$\\
$c_3\cdot x+c_4\cdot y + f_2 = b_2$
\end{quotation}
\end{quotation}
Suponha-se que a base arbitrada inicialmente foi:\\
$(x=0;\quad y=0;\quad f_1=b_1;\quad f_2=b_2)$, tal como recomendado.\\[0.2cm]
Tem-se que $F=v_1\cdot x +v_2\cdot y+0\cdot f_1+0\cdot f_2+0\cdot f_3 + 0$\\
portanto $\quad F-v_1\cdot x -v_2\cdot y-0\cdot f_1-0\cdot f_2-0\cdot f_3 = 0$.\\
Atente-se na negatividade dos coeficientes quando se deixa o \textbf{termo independente} isolado.\\[0.2cm]
\begin{minipage}{\textwidth}
Por uma questão de organização este método é mais simples de ser feito num suporte chamado a "\textbf{tabela do simplex}". 
Começa-se por definir uma coluna para cada variável, e no fim uma coluna de termos independentes.\\
É colocada uma linha por cada variável básica.\\
Os valores a preencher na $n$-ésima linha indicam o coeficiente da variável dessa coluna na $n$-ésima restrição.\\
Por baixo da ultima linha coloca-se a função objetivo e seus coeficientes.\\
\begin{tabular}{l|llll|l}
      & x & y & $f_1$ & $f_2$ & T.I.\\\hline
$f_1$ & $c_1$ & $c_2$ & 1  & 0  & $b_1$\\
$f_2$ & $c_3$ & $c_4$ & 0  & 1  & $b_2$\\\hline
$F$   & $-v_1$ & $-v_2$ & 0  & 0  & $r=0$
\end{tabular}\\
Atente-se na negatividade dos coeficientes da função objetivo, conforme visto acima.\\[0.3cm]
\end{minipage}
A partir daqui o algoritmo consiste na repetição ordenada dos seguintes passos:
\begin{enumerate}
\item \textbf{Escolher uma variável para entrar na base}\\
Pretende-se escolher a variável com maior potencial de crescimento, com o maior coeficiente na função objetivo.\\
Escolhe-se a que tenha o valor \underline{mais negativo} na ultima linha da tabela.\\
Neste caso $min(\{-v_1, -v_2\})$ (ou $max(\{v_1, v_2\})$). A variável no topo dessa coluna é eleita.
\item \textbf{Escolher uma variável para sair da base}\\
Suponha-se que no passo anterior se optou pela 2ª coluna. Para cada $c_i$ nessa linha, divide-se pelo termo independente (T.I.) respetivo. A linha que obtiver o \underline{menor} resultado é a que contem a variável básica que mais limita o crescimento da variável que vai entrar para a base.\\
Supondo que seria a 1ª linha, $f_1$ seria a variável a sair.
\item \textbf{Reconstruir o quadro e verificar se a solução é ótima}\\
Seja $y$ a variável que entrou, e $f_1$ a variável que saiu da base.\\
Para reconstruir a tabela gerando uma nova solução básica, recorre-se ao \textbf{pivô}, isto é, o coeficiente que se encontra na coluna que pertencia a $y$ na tabela anterior. P.ex estando a operar na 3ª linha o pivô é $-v_2$.
\begin{itemize}
\item Coloca-se a linha que pertencia a $f_1$ na tabela anterior.\\
Onde constava $f_1$ na 1ª coluna passa a constar $y$.\\
Os elementos a colocar são os mesmos que estavam nessa linha todos divididos pelo pivô (neste caso $c_2$).\\
Neste caso $[\frac{c_1}{c_2},\frac{c_2}{c_2},\frac{1}{c_2},\frac{0}{c_2},\frac{b_1}{c_2}]$\\
\item Coloca-se as restantes linhas.\\
Multiplica-se a linha obtida no ponto anterior pelo simétrico do pivô da linha a colocar e soma-se os valores que a linha continha.\\
Começa-se pela solução, neste caso a linha será : $-(-v_2) \times[\frac{c_1}{c_2},\frac{c_2}{c_2},\frac{1}{c_2},\frac{0}{c_2},\frac{b_1}{c_2}] + [-v_1,-v_2, 0,0, r]$
\end{itemize}
\item \textbf{Verificar a otimalidade da solução}\\
A solução é ótima quando não existem coeficiêntes negativos na linha da função objetivo.\\
Os valores ótimos das diversas variáveis encontram-se nessa mesma linha.\\
Havendo valores negativos repete-se a partir do 1º passo.\\
\textit{Nota}: É uma boa ideia no passo anterior colocar a ultima linha antes das demais para verificar a otimalidade. (Caso seja solução ótima é escusado calcular as restantes linhas.)
\end{enumerate}
A tabela resultante ao fim de uma iteração (com as variáveis assumidas a título de exemplo) é:\\
\begin{tabular}{l|cccc	|l}
      & x & y & $f_1$ & $f_2$ & T.I.\\\hline
$y$ & $\frac{c_1}{c_2}$ & $\frac{c_2}{c_2}$ & $\frac{1}{c_2}$ & $\frac{0}{c_2}$ & $\frac{b_1}{c_2}$\\
$f_2$ & $(-c_2) \times \frac{c_1}{c_2} + c_3 $ & $(-c_2) \times \frac{c_2}{c_2} + c_4$ & $(-c_2) \times \frac{1}{c_2} + 1$ & $(-c_2) \times \frac{0}{c_2} + 0$ & $(-c_2) \times \frac{b_1}{c_2} + b_1$\\\hline
$F$ & $(-(-v_2)) \times \frac{c_1}{c_2} + c_3 $ & $(-(-v_2)) \times \frac{c_2}{c_2} + c_4$ & $(-(-v_2)) \times \frac{0}{c_2} + 0$ & $(-(-v_2)) \times \frac{1}{c_2} + 0$ & $(-(-v_2)) \times \frac{b_1}{c_2} + r$
\end{tabular}\\
Repare-se que a ordem em que são efetuadas as introduções de linhas facilita muitos dos cálculos.
\section{Variáveis Binárias}
Variáveis binárias são variáveis que só podem tomar os valores $\{0,1\}$. Essas variáveis ajudam a resolver algumas variantes de problemas similares a PL que tem restrições que não são satisfazíveis apenas com minorações e majorações.\\
Uma constante importante que auxilia alguns destes problemas é representada $M$, representa um numero arbitrariamente grande, o suficiente para nunca restringir o problema.
\subsection{Lote Mínimo}
Problemas deste tipo tem variáveis que só se ativam se conseguirem obter um certo valor mínimo.\\
Uma fabrica não abre portas para produzir uma unidade.\\
\begin{tikzpicture}
\draw[->] (0,0) -- (10,0) node[anchor=north]{$X$};
\draw	(0,0) node[anchor=north] {0}
		(3,0) node[anchor=north] {$k$};
\draw[dotted] (3,0) -- (3,1.5);
\draw[thick] (3,0) -- (3,1) -- (10,1);
\end{tikzpicture}\\
Se $Z$ for uma variável binária, então este problema resolve-se com as restrições:
\begin{quotation}
\noindent $X \geq k \cdot Z$\\
$X \leq M \cdot Z$\\
$Z \in \{0,1\}$\\
$X \geq 0$
\end{quotation}
Quando a variável binária for falsa ($=0$), vai majorar $X$ com $0$.\\
$X \leq M \cdot 0 \Leftrightarrow X \leq 0$\\
Dado que $X$ é estritamente positiva, nesse caso $X=0$.
\subsection{Custo inicial}
Para se produzir, frequentemente existem custos iniciais que não contribuem diretamente para o produzido (aluguer do espaço, compra das maquinas).\\
Enquanto que um problema de PL normal costuma ser visto como o gráfico da esquerda (em que $C=$custo, $P=$produzido), nestas situações o gráfico da direita é mais realista:\\
\begin{tikzpicture}
\draw[->] (0,0) -- (0,3) node[anchor=east] {$C$};
\draw[->] (0,0) -- (8,0) node[anchor=north]{$P$};
\draw	(0,0) node[anchor=north] {0};
\draw[thick] (0,0) -- (8,2);
\end{tikzpicture}
\begin{tikzpicture}
\draw[->] (0,0) -- (0,3) node[anchor=east] {$C$};
\draw[->] (0,0) -- (8,0) node[anchor=north]{$P$};
\draw	(0,0) node[anchor=north] {0}
		(0,1) node[anchor=east] {$c_i$};
\draw[dotted] (0,1) -- (8,1);
\draw[thick] (0,1) -- (8,3);
\end{tikzpicture}\\
Para estas situações é utilizada uma variável binária que indica se chegou a existir produção dessa variável.\\
Essa variável é colocada na função objetivo aliada á constante de custo inicial ($c_i$ no gráfico).
\begin{quotation}
\noindent $Min\quad F= \dots + c_i \cdot Z + c_k \cdot X_i + \dots$\\
$X_i \leq M \cdot Z$\\
$Z \in \{0,1\}$
\end{quotation}
Quando a variável binária é falsa ($=0$) não se produz nada, e $X_i \leq 0$.\\
Caso contrario $M$ não limita $X_i$ e a constante de custo inicial é adicionada á função objetivo.
\subsection{Variável discreta}
Quando se tem uma variável que só pode tomar valores de um conjunto discreto, necessita-se de tantas variáveis binárias quão o tamanho do conjunto.\\
Esse problema é formulado:
\begin{quotation}
\noindent
$X \in \{x_1, x_2, \dots, x_n\}$\\
$X=\dots + Z_1\cdot x_1 + Z_2 \cdot x_2 + \dots + Z_n \cdot x_n + \dots$\\
$Z_1 + Z_2 + \dots + Z_n = 1$\\
$Z_1, Z_2, \dots, Z_n \in \{0,1\}$
\end{quotation}
Atente-se que apenas um dos $Z$s pode ser positivo.
\subsection{Função objetivo com inclinação variável}
Caso em que a função objetivo contem uma variável cuja inclinação varia em função do produzido.\\
Utilizado em cenários de rendimentos decrescentes (por exemplo o preço por fotocópia descer quantas mais páginas se imprimam), mas não só, não havendo de todo obrigatoriedade de haver um comportamento monótono.\\
\begin{tikzpicture}
\draw[->] (0,0) -- (0,3) node[anchor=east] {$C$};
\draw[->] (0,0) -- (9,0) node[anchor=north]{$P$};
\draw	(0,0) node[anchor=north] {0}
		(1,0) node[anchor=north] {$a$}
		(4,0) node[anchor=north] {$b$}
		(8,0) node[anchor=north] {$c$};
\draw[dotted] (1,0) -- (1,1) ;
\draw[dotted] (4,0) -- (4,1.5) ;
\draw[dotted] (8,0) -- (8,3) ;
\draw[thick] (0,0) -- (1,1) -- (4,1.5) -- (8,3);
\end{tikzpicture}\\
Sejam $d_1, d_2, \dots, d_n$ os declives dos sucessivos segmentos (entre [$a,b$], entre [$b,c$], ...).\\
Decompõe-se $X$ em $n$ sub-variáveis ($Y_1, \dots, Y_n$), uma para cada um dos $n$ sub-intervalos (a figura tinha 3).\\
$0 \leq Y_1 \leq a;\quad 0 \leq Y_2 \leq b-a;\quad \dots \quad  0 \leq Y_n \leq n-m$.\\
Formula-se o problema:
\begin{quotation}
\noindent$X = Y_1, Y_2 + \dots + Y_n$\\
$F = d_1 \cdot Y_1 + d_2 \cdot Y_2 + \dots + d_n \cdot Y_n$\\
$a \cdot Z_1 \leq Y_1 \leq a$\\
$b \cdot Z_2 \leq Y_2 \leq b-a \cdot Z_1$\\
$\vdots$\\
$0 \leq Y_n \leq n-m \cdot Z_{n-1}$\\
$Z_1, \dots, Z_{n-1} \in \{0,1\}$
\end{quotation}
A função objetivo é o declive multiplicado pela amplitude o intervalo correspondente.\\
As variáveis binárias obrigam os segmentos a ser usados consecutivamente, pela ordem em que aparecem, até ao ultimo segmento utilizado.
\chapter{Teoria da Decisão}
\chapter{Teoria das Filas de Espera}
\chapter{Simulação}
\chapter{Programação Linear}
\end{document}
